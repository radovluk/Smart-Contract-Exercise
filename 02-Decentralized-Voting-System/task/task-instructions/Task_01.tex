\documentclass[12pt]{article}
\usepackage[utf8]{inputenc}
\usepackage{hyperref}
\usepackage{listings}
\usepackage{xcolor}
\usepackage{geometry}

\usepackage{minted} % For advanced code listings

% Define a custom minted style (optional)
\usemintedstyle{colorful} % You can choose from various styles like 'monokai', 'tango', 'colorful', etc.

% Custom color setup
\definecolor{bashtextcolor}{RGB}{0, 0, 0} % Define black color

% Define a new command for inline code using minted
\newcommand{\codeinline}[1]{\mintinline{text}{#1}}

\geometry{a4paper, margin=1in}

\title{Smart Contracts Exercise 02: \\ Decentralized Voting System}
\author{}
\date{}

% Define a new command for inline code with a dark background
\newcommand{\codeblack}[1]{%
  \texttt{\colorbox{black!7}{\textcolor{black}{#1}}}%
}

% Define a new command for inline code with a dark background
\newcommand{\codegrey}[1]{%
  \texttt{\colorbox{black!4}{\textcolor{black}{#1}}}%
}

% Define custom colors (optional)
\definecolor{myURLColor}{RGB}{0, 102, 204} % Example: A shade of blue

\hypersetup{
    colorlinks=true,        % Enable colored links
    linkcolor=blue,         % Color for internal links (e.g., \ref, \cite)
    citecolor=blue,         % Color for citations
    filecolor=magenta,      % Color for file links
    urlcolor=myURLColor     % Color for external URLs
}

% Define a style for code listings
\lstdefinestyle{mystyle}{
    backgroundcolor=\color{lightgray!20},   
    commentstyle=\color{green!50!black},
    keywordstyle=\color{blue},
    numberstyle=\tiny\color{gray},
    stringstyle=\color{red},
    basicstyle=\ttfamily\footnotesize,
    breakatwhitespace=false,         
    breaklines=true,                 
    captionpos=b,                    
    keepspaces=true,                 
    numbers=left,                    
    numbersep=5pt,                  
    showspaces=false,                
    showstringspaces=false,
    showtabs=false,                  
    tabsize=2
}

\lstset{style=mystyle}
% Adding package for header and footer
\usepackage{fancyhdr}
\pagestyle{fancy}

% Define header and footer
\fancyhf{} % Clear current settings
\fancyhead[L]{Smart Contracts Exercise 02} % Left header
\fancyhead[R]{\thepage} % Right header with page number

\renewcommand{\headrulewidth}{0.4pt} % Line below header
% \renewcommand{\footrulewidth}{0.4pt} % Line above footer

\begin{document}

\maketitle
\section{Introduction}

In this exercise, you will implement a smart contract of a decentralized voting system on the blockchain. The goal of this exercise is to familiarize yourself with the basics of the Solidity language.

\subsection{Prerequisites}

Ensure that you have already installed the following on your system:

\begin{itemize}
    \item \textbf{Node.js} - \url{https://nodejs.org/en/}
    An open-source, cross-platform, back-end JavaScript runtime environment that runs on the V8 engine and executes JavaScript code outside a web browser. 
    \item \textbf{NPM}: Node Package Manager, which comes with Node.js.
\end{itemize}

Open your terminal and run the following commands to verify the installations:

\begin{minted}[bgcolor=gray!5, fontsize=\footnotesize]{bash}
$ node -v
$ npm -v
\end{minted}

Both commands should return the installed version numbers of Node.js and NPM respectively. Node.js provides the runtime environment required to execute JavaScript-based tools like Hardhat, while NPM is used to manage the packages and dependencies needed for development. It is recommended that you use NPM 7 or higher.

\subsection{Project Set Up}

To get started, visit the following \href{https://gitlab.fel.cvut.cz/radovluk/smart-contracts-exercises/-/tree/main/02%20-%20Decentralized%20Voting%20System/task/task%20code}{GitLab repository} and clone it to your local machine. This repository contains a template in which you will complete this exercise. After you clone the repository start with the following command within your project folder:

\begin{minted}[bgcolor=gray!5, fontsize=\footnotesize]{bash}
  $ npm install
\end{minted}
This will install all the necessary dependencies for the project. Your implementation will be in the file \textit{contracts/Voting.sol}. In this file, there are \#TODO comments where you should implement the required functionality. To fulfill this task you need to pass all the provided tests. You can run the tests with the following command:

\begin{minted}[bgcolor=gray!5, fontsize=\footnotesize]{bash}
  $ npx hardhat test
\end{minted}

There is also a deployment script in the \textit{scripts} folder. You can deploy the contract to the local hardhat network with the following command:
\begin{minted}[bgcolor=gray!5, fontsize=\footnotesize]{bash}
  $ npx hardhat run scripts/deploy.js
\end{minted}

\section{Specification: Voting Contract}

\subsection{Overview}

The \textbf{Voting} contract is a simple implementation of a voting system using Solidity. It allows the contract owner to add candidates, and any address to vote exactly once for a candidate. The contract includes the following functionalities:
\begin{itemize}
    \item The contract owner can add candidates.
    \item Any address can vote exactly once for a candidate.
    \item The contract tracks the number of votes each candidate has received.
    \item The contract tracks whether an address has already voted.
    \item A function to get the total number of candidates.
    \item A function to retrieve a candidate's name and vote count by index.
    \item A function to get the index of the winning candidate.
\end{itemize}

\subsection{Solidity Crash Course}

The \textbf{Voting} contract is designed to facilitate a decentralized voting system. Below are some solidity code snippets that you might find useful for implementing the contract and explanations of the key components.

\subsubsection{State Variables}

\textbf{State variables} are used to store data permanently on the blockchain. They represent the contract's state and can be accessed and modified by the contract's functions.

\begin{minted}[bgcolor=gray!5, fontsize=\footnotesize]{solidity}
  // Address of the contract owner
  address public owner;

  // Dynamic array to store all candidates
  Candidate[] public candidates;

  // Mapping to track whether an address has already voted
  mapping(address => bool) public hasVoted;
\end{minted}

\subsubsection{Structs}

\textbf{Structs} are custom data types that allow you to group related data together. They are useful for organizing complex data structures within the contract.

\begin{minted}[bgcolor=gray!5, fontsize=\footnotesize]{solidity}
  /**
  * @dev Struct to represent a candidate.
  * @param name The name of the candidate.
  * @param voteCount The number of votes the candidate has received.
  */
 struct Candidate {
     string name;
     uint voteCount;
  }
\end{minted}

\subsubsection{Modifiers}

\textbf{Modifiers} are used to change the behavior of functions in a declarative way. They can enforce rules or conditions before executing a function's code.

\begin{minted}[bgcolor=gray!5, fontsize=\footnotesize]{solidity}
  // Modifier to restrict access to the contract owner
  modifier onlyOwner() {
      require(msg.sender == owner, "Only the owner can call this function");
      _; // Continue executing the function
  }

  function addCandidate(string memory _name) public onlyOwner {
    // Only the contract owner can call this function
  }
\end{minted}

\subsubsection{Functions}

\textbf{Functions} define the behavior of the contract. They can read and modify the contract's state, perform computations, and interact with other contracts or external systems.

\subsubsection{Events}

\textbf{Events} are used to log information on the blockchain that can be accessed by off-chain applications. They are essential for tracking contract activities and facilitating interactions with the user interface.

\begin{minted}[bgcolor=gray!5, fontsize=\footnotesize]{solidity}
  /**
  * @dev Event emitted when a vote is cast.
  * @param voter The address of the voter.
  * @param candidateIndex The index of the candidate voted for.
  */
 event Voted(address indexed voter, uint indexed candidateIndex);

 /**
  * @dev Event emitted when a new candidate is added.
  * @param name The name of the candidate to be added.
  */
 event CandidateAdded(string name);
\end{minted}

\subsubsection{Functionality Provided by Solidity}
Here are some useful code snippets you might need:

\begin{minted}[bgcolor=gray!5, fontsize=\footnotesize]{solidity}
// Sender of the transaction
address sender = msg.sender;

// Amount sent with the transaction
uint amount = msg.value;

// Enforcing conditions
require(condition, "Error message");

// Casting arbitrary data to uint
uint number = uint(data);

// Empty address
address emptyAddress = address(0);

// Emit an event
emit EventName(parameters);
\end{minted}


\end{document}
