\documentclass[12pt]{article}
\usepackage[utf8]{inputenc}
\usepackage{hyperref}
\usepackage{xcolor}
\usepackage{geometry}
\usepackage{listings-solidity}  % Include code highlighting
\usepackage{listings}
\usepackage{enumitem}
\usepackage{minted}

% Define a custom minted style
\usemintedstyle{colorful}

% Set smaller bullets for all itemize environments
\setlist[itemize]{label=\small\textbullet}

% Custom color setup
\definecolor{bashtextcolor}{RGB}{0, 0, 0} % Define black color

% Define custom colors (optional)
\definecolor{myURLColor}{RGB}{0, 102, 204} % Example: A shade of blue

\geometry{a4paper, margin=1in}

\title{Smart Contracts Exercise 02: \\ Decentralized Voting System}
\author{}
\date{}

% Define a new command for inline code with a dark background
\newcommand{\codeblack}[1]{%
  \texttt{\colorbox{black!7}{\textcolor{black}{#1}}}%
}

% Define a new command for inline code with a dark background
\newcommand{\codegrey}[1]{%
  \texttt{\colorbox{black!4}{\textcolor{black}{#1}}}%
}

\hypersetup{
    colorlinks=true,        % Enable colored links
    linkcolor=blue,         % Color for internal links (e.g., \ref, \cite)
    citecolor=blue,         % Color for citations
    filecolor=magenta,      % Color for file links
    urlcolor=myURLColor     % Color for external URLs
}

% Define a style for code listings
\lstdefinestyle{mystyle}{
    backgroundcolor=\color{lightgray!20},   
    commentstyle=\color{green!50!black},
    keywordstyle=\color{blue},
    numberstyle=\tiny\color{gray},
    stringstyle=\color{red},
    basicstyle=\ttfamily\footnotesize,
    breakatwhitespace=false,         
    breaklines=true,                 
    captionpos=b,                    
    keepspaces=true,                 
    numbers=left,                    
    numbersep=5pt,                  
    showspaces=false,                
    showstringspaces=false,
    showtabs=false,                  
    tabsize=2
}

\lstset{style=mystyle}

% Adding package for header and footer
\usepackage{fancyhdr}
\pagestyle{fancy}

% Define header and footer
\fancyhf{} % Clear current settings
\fancyhead[L]{Smart Contracts Exercise 02} % Left header
\fancyhead[R]{\thepage} % Right header with page number

\renewcommand{\headrulewidth}{0.4pt} % Line below header
% \renewcommand{\footrulewidth}{0.4pt} % Line above footer

\begin{document}

\maketitle
\section{Introduction}

In this exercise, you will implement a smart contract for a decentralized voting system on the blockchain. The goal of this exercise is to familiarize yourself with the basics of the Solidity programming language.

\subsection*{Project Setup}

You have two options for working with this exercise: using a Docker container or local installation. Choose the option that best fits your preferences.

\subsection{Using Docker with VS Code}

This option uses Docker to create a development environment with all necessary tools and dependencies preinstalled.

\subsubsection*{Prerequisites:}

\begin{itemize}
    \item \textbf{\href{https://www.docker.com/products/docker-desktop}{Docker}} - A platform for developing, shipping, and running applications in containers.
    \item \textbf{\href{https://code.visualstudio.com/}{Visual Studio Code}} - A lightweight yet powerful source code editor.
    \item \textbf{\href{https://marketplace.visualstudio.com/items?itemName=ms-vscode-remote.remote-containers}{Dev Containers}} - An extension to VS Code that lets you use a Docker container as a full-featured development environment.
\end{itemize}

\subsubsection*{Setting Up the Project:}

\begin{enumerate}
  \item Visit the following \href{https://github.com/radovluk/Smart-Contract-Exercise/tree/main/02-Decentralized-Voting-System/task/task-code}{GitHub repository} and clone it to your local machine.
  \item Open the repository folder in VS Code.
  \item When prompted, click ``Reopen in Container'' or use the command palette (F1) and run \codegrey{Dev Containers: Reopen in Container}.
\end{enumerate}

\subsection{Local Setup}

If you prefer working directly on your machine without Docker, you can set up the development environment locally.

\subsubsection*{Prerequisites}
\begin{itemize}
    \item \textbf{Node.js}: \url{https://nodejs.org/en/} - An open-source, cross-platform, back-end JavaScript runtime environment that runs on the V8 engine and executes JavaScript code outside a web browser.
    \item \textbf{NPM}: Node Package Manager, which comes with Node.js.
\end{itemize}

\noindent
Open your terminal and run the following commands to verify the installations:

\begin{minted}[bgcolor=gray!5, fontsize=\footnotesize]{bash}
$ node -v
$ npm -v
\end{minted}

Both commands should return the installed version numbers of Node.js and NPM respectively. Node.js provides the runtime environment required to execute JavaScript-based tools like Hardhat, while NPM is used to manage the packages and dependencies needed for development.

\subsubsection*{Setting Up the Project}

\begin{enumerate}
    \item Visit the following \href{https://github.com/radovluk/Smart-Contract-Exercise/tree/main/02-Decentralized-Voting-System/task/task-code}{GitHub repository} and clone it to your local machine.
    \item Open a terminal and navigate to the project directory.
    \item Install the project dependencies by running \codegrey{npm install}.
\end{enumerate}

\section{Task Specification: Voting Contract}

Your implementation will be in the file \texttt{contracts/Voting.sol}. In this file, there are \#TODO comments where you should implement the required functionality. To complete this task, you need to pass all the provided tests. You can run the tests with the following command:

\begin{minted}[bgcolor=gray!5, fontsize=\footnotesize]{bash}
  $ npx hardhat test
\end{minted}

There is also a deployment script in the \texttt{scripts} folder. You can deploy the contract to the local Hardhat network with the following command:

\begin{minted}[bgcolor=gray!5, fontsize=\footnotesize]{bash}
  $ npx hardhat run scripts/deploy.js
\end{minted}

\subsection{Overview}

The \textbf{Voting} contract is a simple implementation of a voting system using Solidity. It allows the contract owner to add candidates and enables any address to vote exactly once for a candidate. The contract includes the following functionality:
\begin{itemize}
    \item The contract owner can add candidates.
    \item Any address can vote exactly once for a candidate.
    \item The contract tracks the number of votes each candidate has received.
    \item The contract tracks whether an address has already voted.
    \item A function to get the total number of candidates.
    \item A function to retrieve a candidate's name and vote count by index.
    \item A function to get the index of the winning candidate.
\end{itemize}

\subsection{Solidity Crash Course}

The \textbf{Voting} contract is designed to facilitate a decentralized voting system. Below are some Solidity code snippets that you might find useful for implementing the contract, along with explanations.

\subsubsection*{State Variables}

State variables are used to store data permanently on the blockchain. They represent the contract's state and can be accessed and modified by the contract's functions.

\noindent
\begin{minipage}[c]{\textwidth}
\begin{lstlisting}[language=Solidity]
  // Address of the contract owner
  address public owner;

  // Dynamic array to store all candidates
  Candidate[] public candidates;

  // Mapping to track whether an address has already voted
  mapping(address => bool) public hasVoted;
\end{lstlisting}
\end{minipage}

\subsubsection*{Structs}

Structs are custom data types that allow you to group related data together. They are useful for organizing complex data structures within the contract.

\noindent
\begin{minipage}[c]{\textwidth}
\begin{lstlisting}[language=Solidity]
  /**
  * @dev Struct to represent a candidate.
  * @param name The name of the candidate.
  * @param voteCount The number of votes the candidate has received.
  */
 struct Candidate {
     string name;
     uint voteCount;
  }
\end{lstlisting}
\end{minipage}

\subsubsection*{Constructor}

The constructor is a special function that runs once during the contract's deployment and cannot be invoked afterward.

\noindent
\begin{minipage}[c]{\textwidth}
\begin{lstlisting}[language=Solidity]
  constructor() {
    // The deployer of the contract is the owner
    owner = msg.sender;
  }
\end{lstlisting}
\end{minipage}

\subsubsection*{Events}

Events are used to log information on the blockchain that can be accessed by off-chain applications. They are essential for tracking contract activities and facilitating interactions with the user interface.

\noindent
\begin{minipage}[c]{\textwidth}
\begin{lstlisting}[language=Solidity]
  /**
  * @dev Event emitted when a vote is cast.
  * @param voter The address of the voter.
  * @param candidateIndex The index of the candidate voted for.
  */
 event Voted(address indexed voter, uint indexed candidateIndex);

 /**
  * @dev Event emitted when a new candidate is added.
  * @param name The name of the candidate to be added.
  */
 event CandidateAdded(string name);
\end{lstlisting}
\end{minipage}

\subsubsection*{Errors}

Errors allow developers to provide more information to the caller about why a condition or operation failed. Errors are used together with the revert statement. Upon failure, they abort and revert \textbf{all} changes made by the transaction.

\noindent
\begin{minipage}[c]{\textwidth}
\begin{lstlisting}[language=Solidity]
  /// Only the owner can call this function.
  error NotOwner();
  /// The candidate name cannot be empty.
  error EmptyCandidateName();

  // revert if condition is not met
  if (msg.sender != owner) revert NotOwner();

  // revert statement
  revert EmptyCandidateName();
\end{lstlisting}
\end{minipage}

\subsubsection*{Modifiers}

Modifiers are used to change the behavior of functions in a declarative way. They can enforce rules or conditions before executing a function's code.

\noindent
\begin{minipage}[c]{\textwidth}
\begin{lstlisting}[language=Solidity]
  // Modifier to restrict access to the contract owner
  modifier onlyOwner() {
      if (msg.sender != owner) revert NotOwner();
      _; // Continue executing the function
  }

  function addCandidate(string memory name) public onlyOwner {
    // Only the contract owner can call this function
  }
\end{lstlisting}
\end{minipage}

\subsubsection*{Functions}

Functions define the behavior of the contract. They can read and modify the contract's state, perform computations, and interact with other contracts or external systems.

\subsubsection*{Useful Code Snippets}
Here are some useful code snippets you might need:

\noindent
\begin{minipage}[c]{\textwidth}
\begin{lstlisting}[language=Solidity]
// Sender of the transaction
address sender = msg.sender;

// Amount sent with the transaction
uint amount = msg.value;

// Casting arbitrary data to uint
uint number = uint(data);

// Empty address
address emptyAddress = address(0);

// Emit an event
emit EventName(parameters);
\end{lstlisting}
\end{minipage}

\pagebreak
\section{More Solidity Concepts}

This extra section covers additional Solidity concepts that are useful for
developing smart contracts, including more detailed function examples,
visibility, and advanced data types. It is not needed to complete the exercise.

\subsubsection*{Function Types and Visibility}
Functions in Solidity can have different visibility modifiers that determine how and from where they can be called:

\noindent
\begin{minipage}[c]{\textwidth}
  \begin{lstlisting}[language=Solidity]
// Public functions can be called internally or via messages
function publicFunction() public returns (uint) {
  return 1;
}
// Private functions can only be called from within this contract
function privateFunction() private returns (uint) {
  return 2;
}
// Internal functions can be called internally or by derived contracts
function internalFunction() internal returns (uint) {
  return 3;
}
// External functions can only be called from other contracts
function externalFunction() external returns (uint) {
  return 4;
}
// View functions promise not to modify the state
function viewFunction() public view returns (uint) {
  return someStateVariable;
}
// Pure functions promise not to modify or read from the state
function pureFunction(uint a, uint b) public pure returns (uint) {
  return a + b;
}
\end{lstlisting}
\end{minipage}

\subsubsection*{Function Modifiers with Parameters}

Modifiers can also accept parameters, making them more flexible:

\noindent
\begin{minipage}[c]{\textwidth}
  \begin{lstlisting}[language=Solidity]
// Modifier with parameters
modifier onlyRole(bytes32 role) {
  require(hasRole(role, msg.sender), "Caller does not have the required role");
  _;
}
// Using the modifier with different roles
function adminFunction() public onlyRole(ADMIN_ROLE) {
// Only addresses with admin role can call this
}
function moderatorFunction() public onlyRole(MODERATOR_ROLE) {
// Only addresses with moderator role can call this
}
\end{lstlisting}
\end{minipage}

\noindent
\begin{minipage}[c]{\textwidth}
  \subsubsection*{Advanced Data Structures}

  Solidity supports several advanced data structures that help organize complex
  data:

  \begin{lstlisting}[language=Solidity]
// Nested mappings for complex relationships
mapping(address => mapping(uint => bool)) public userPermissions;
// Set a permission
function setPermission(address user, uint permissionId, bool value) public {
  userPermissions[user][permissionId] = value;
}
// Arrays with push and pop operations
uint[] public values;
function addValue(uint value) public {
  values.push(value);
}
function removeLastValue() public {
  values.pop();
}
// Enums for named constants
enum Status { Pending, Active, Inactive, Completed }
Status public currentStatus;
function setStatus(Status newStatus) public {
  currentStatus = newStatus;
}
\end{lstlisting}

  \subsubsection*{Memory Management in Solidity}

  \begin{lstlisting}[language=Solidity]
// Storage - persisted between function calls (expensive)
uint[] public storageArray;
// Memory - temporary during function execution (cheaper)
function processArray(uint[] memory memoryArray) public {
  // This array exists only during function execution
  uint[] memory tempArray = new uint[](memoryArray.length);
  
  for(uint i = 0; i < memoryArray.length; i++) {
    tempArray[i] = memoryArray[i] * 2;
  }

  // Store results in contract storage if needed
  for(uint i = 0; i < tempArray.length; i++) {
    storageArray.push(tempArray[i]);
  }
}
// Calldata - read-only, non-modifiable location (most gas efficient)
function readOnlyProcess(uint[] calldata calldataArray) external {
  // Can read from but not modify calldataArray
  for(uint i = 0; i < calldataArray.length; i++) {
    // Process without modifying
  }
}
\end{lstlisting}
\end{minipage}

\noindent
\begin{minipage}[c]{\textwidth}
  \subsubsection*{Error Handling}
  Solidity provides several mechanisms for error handling:

  \begin{lstlisting}[language=Solidity]
// Using require for input validation
function transfer(address to, uint amount) public {
  require(to != address(0), "Cannot transfer to zero address");
  require(amount > 0, "Amount must be greater than zero");
  require(balances[msg.sender] >= amount, "Insufficient balance");
  
  balances[msg.sender] -= amount;
  balances[to] += amount;
}

// Using assert for internal consistency checks
function internalOperation(uint a, uint b) private {
  uint result = a + b;
  // Should never happen if our code is correct
  assert(result >= a && result >= b); // Check for overflow
}

// Custom errors (more gas efficient than string messages)
error InsufficientBalance(address user, uint available, uint required);

function withdraw(uint amount) public {
  if (balances[msg.sender] < amount) {
    revert InsufficientBalance(msg.sender, balances[msg.sender], amount);
  }
  
  balances[msg.sender] -= amount;
  payable(msg.sender).transfer(amount);
}
\end{lstlisting}
\end{minipage}

\subsubsection*{Gas Optimization Techniques}
Gas optimization is crucial for developing cost-effective smart contracts:

\noindent
\begin{minipage}[c]{\textwidth}
  \begin{lstlisting}[language=Solidity]
// Use uint256 instead of smaller sizes (usually)
uint256 public largeNumber; // Often cheaper than uint8, uint16, etc.

// Pack variables that are used together
struct PackedData {
  uint128 firstValue;  // These two uint128 variables
  uint128 secondValue; // will fit in a single storage slot
}

// Cache array length in loops
function processItems() public {
  uint length = items.length; // Read once
  for(uint i = 0; i < length; i++) {
    // Process items[i]
  }
}

// Use calldata for read-only function parameters
function readOnlyOperation(string calldata text) external pure returns (uint) {
  return bytes(text).length;
}
\end{lstlisting}
\end{minipage}

\subsubsection*{Events and Logging}
Events are crucial for off-chain services and DApp frontends, which can listen for events and react to changes on the blockchain.

\noindent
\begin{minipage}[c]{\textwidth}
  \begin{lstlisting}[language=Solidity]
// Simple event with basic data
event Transfer(address indexed from, address indexed to, uint amount);
// Event with additional data
event VoteCast(
  address indexed voter,
  uint indexed candidateId,
  uint timestamp,
  string comments
);
function castVote(uint candidateId, string memory comments) public {
  // Process vote...
  
  // Emit event for off-chain applications to track
  emit VoteCast(msg.sender, candidateId, block.timestamp, comments);
}
// Events for multi-step processes
event ProcessStarted(uint indexed processId, address initiator);
event ProcessStep(uint indexed processId, uint step, string description);
event ProcessCompleted(uint indexed processId, bool success);
function runProcess(uint processId) public {
  emit ProcessStarted(processId, msg.sender);
  
  // Step 1
  emit ProcessStep(processId, 1, "Validation");
  // ... processing ...

  // Step 2
  emit ProcessStep(processId, 2, "Calculation");
  // ... processing ...

  emit ProcessCompleted(processId, true);
}
\end{lstlisting}
\end{minipage}

\medskip
\noindent
Off-chain services or DApp frontend code:

\noindent
\begin{minipage}[c]{\textwidth}
  \begin{minted}[bgcolor=gray!5, fontsize=\footnotesize]{javascript}
// Import the ethers.js library
const { ethers } = require("ethers");
// Connect to an Ethereum node (replace with your provider URL)
const provider = new ethers.providers
  .JsonRpcProvider("https://mainnet.infura.io/v3/YOUR_KEY");
// Define minimal ABI with just the event we want to listen for
const abi = ["event VoteCast(address indexed voter, uint indexed candidateId, \
  uint timestamp, string comments)"];
// Create contract instance with address, ABI and provider
const contract = new ethers.Contract("0xContractAddress", abi, provider);

// Listen for vote events - the callback receives all parameters defined in the event
contract.on("VoteCast", (voter, candidateId, timestamp, comments) => {
  console.log(`Vote from ${voter} for candidate ${candidateId}: 
  "${comments}" at ${new Date(timestamp * 1000)}`);
});
  \end{minted}
\end{minipage}

\noindent
\begin{minipage}[c]{\textwidth}
  \subsubsection*{Inheritance and Contract Interaction}
  Solidity supports contract inheritance and interfaces:

  \begin{lstlisting}[language=Solidity]
// Base contract
contract Ownable {
  address public owner;

  constructor() {
    owner = msg.sender;
  }

  modifier onlyOwner() {
    require(msg.sender == owner, "Not owner");
    _;
  }

  function transferOwnership(address newOwner) public onlyOwner {
    require(newOwner != address(0), "New owner cannot be zero address");
    owner = newOwner;
  }
}
// Derived contract - inherits all functions and modifiers from Ownable
contract VotingEnhanced is Ownable {
  function addCandidate(string memory name) public onlyOwner {
    // Only the owner can add candidates
  }
}
// Interface definition
interface IVoting {
  function vote(uint candidateId) external;
  function getCandidateCount() external view returns (uint);
  function getCandidate(uint index) external view returns (string memory name, uint voteCount);
}
// Using another contract
contract VotingClient {
  IVoting public votingContract;
  
  constructor(address votingAddress) {
    votingContract = IVoting(votingAddress);
  }

  function voteForCandidate(uint candidateId) public {
    votingContract.vote(candidateId);
  }

  function getNumberOfCandidates() public view returns (uint) {
    return votingContract.getCandidateCount();
  }
}
\end{lstlisting}
\end{minipage}

\noindent
This example demonstrates inheritance through a base contract (Ownable) that's extended by a derived contract (VotingEnhanced), interface definition (IVoting) for standardizing contract interactions, and contract-to-contract communication through interfaces.

\medskip
\noindent
To see more advanced smart contract examples, visit the \href{https://docs.soliditylang.org/en/latest/solidity-by-example.html}{Solidity by Example} section of the Solidity documentation.

\end{document}
