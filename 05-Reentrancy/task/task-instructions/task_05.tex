\documentclass[12pt]{article}
\usepackage[utf8]{inputenc}
\usepackage{hyperref}
\usepackage{listings}
\usepackage{xcolor}
\usepackage{geometry}
\usepackage{graphicx} % For including graphics
\usepackage{minted} % For advanced code listings
\usepackage{listings-solidity}  % Include Solidity highlighting

% Define a custom minted style (optional)
\usemintedstyle{colorful} % You can choose from various styles like 'monokai', 'tango', 'colorful', etc.

% Custom color setup
\definecolor{bashtextcolor}{RGB}{0, 0, 0} % Define black color

% Define a new command for inline code using minted
\newcommand{\codeinline}[1]{\mintinline{text}{#1}}

\geometry{a4paper, margin=1in}

\title{Smart Contracts Exercise 05: \\ Reentrancy}
\author{}
\date{}

% Define a new command for inline code with a dark background
\newcommand{\codeblack}[1]{%
  \texttt{\colorbox{black!7}{\textcolor{black}{#1}}}%
}

% Define a new command for inline code with a dark background
\newcommand{\codegrey}[1]{%
  \texttt{\colorbox{black!4}{\textcolor{black}{#1}}}%
}

% Define custom colors (optional)
\definecolor{myURLColor}{RGB}{0, 102, 204} % Example: A shade of blue

\hypersetup{
    colorlinks=true,        % Enable colored links
    linkcolor=blue,         % Color for internal links (e.g., \ref, \cite)
    citecolor=blue,         % Color for citations
    filecolor=magenta,      % Color for file links
    urlcolor=myURLColor     % Color for external URLs
}

% Define a style for code listings
\lstdefinestyle{mystyle}{
    backgroundcolor=\color{lightgray!20},   
    commentstyle=\color{green!50!black},
    keywordstyle=\color{blue},
    numberstyle=\tiny\color{gray},
    stringstyle=\color{red},
    basicstyle=\ttfamily\footnotesize,
    breakatwhitespace=false,         
    breaklines=true,                 
    captionpos=b,                    
    keepspaces=true,                 
    numbers=left,                    
    numbersep=5pt,                  
    showspaces=false,                
    showstringspaces=false,
    showtabs=false,                  
    tabsize=2
}

\lstset{style=mystyle}
% Adding package for header and footer
\usepackage{fancyhdr}
\pagestyle{fancy}

% Define header and footer
\fancyhf{} % Clear current settings
\fancyhead[L]{Smart Contracts 05} % Left header
\fancyhead[R]{\thepage} % Right header with page number

\renewcommand{\headrulewidth}{0.4pt} % Line below header
% \renewcommand{\footrulewidth}{0.4pt} % Line above footer

\begin{document}

\maketitle
\section{Introduction}
TODO: Write some introduction

\subsection*{Prerequisites}

Ensure that you have already installed the following on your system:

\begin{itemize}
    \item \textbf{Node.js} - \url{https://nodejs.org/en/}
    An open-source, cross-platform, back-end JavaScript runtime environment that runs on the V8 engine and executes JavaScript code outside a web browser. 
    \item \textbf{NPM}: Node Package Manager, which comes with Node.js.
\end{itemize}

Open your terminal and run the following commands to verify the installations:

\begin{minted}[bgcolor=gray!5, fontsize=\footnotesize]{bash}
$ node -v
$ npm -v
\end{minted}

Both commands should return the installed version numbers of Node.js and NPM respectively. Node.js provides the runtime environment required to execute JavaScript-based tools like Hardhat, while NPM is used to manage the packages and dependencies needed for development. It is recommended that you use NPM 7 or higher.

For the purposes of this exercise, you will need an Infura API key and a configured wallet. If you do not have this set up yet, we recommend going through the Smart Contracts Exercise 01: Hello, Blockchain World! where everything is explained. Ensure that configuration variables are set for your Hardhat projects. You can verify this by running:

\begin{minted}[bgcolor=gray!5, fontsize=\footnotesize]{bash}
$ npx hardhat vars get INFURA_API_KEY
$ npx hardhat vars get SEPOLIA_PRIVATE_KEY
\end{minted}

\subsection*{Project Set Up}

To get started, visit the following \href{https://gitlab.fel.cvut.cz/radovluk/smart-contracts-exercises/-/tree/main/05-Reentrancy/task/task-code}{GitLab repository} and clone it to your local machine. This repository contains a template in which you will complete this exercise. After you clone the repository, start with the following command within your project folder:

\begin{minted}[bgcolor=gray!5, fontsize=\footnotesize]{bash}
$ npm install
\end{minted}

\end{document}
